\documentclass{ctexart}
\usepackage[AutoFallBack=true]{xeCJK}
	\setCJKmainfont{Source Han Serif SC}
	\setCJKfallbackfamilyfont{rm}{MS Mincho}
\usepackage{geometry}
	\geometry{left=1in,right=1in,top=1in,bottom=1in}
\usepackage{titling}
\usepackage{enumitem}
	\setlist{noitemsep,nosep}
\usepackage[colorlinks,linkcolor=blue]{hyperref}
\usepackage[symbol=$\dagger$,numberlinked=false]{footnotebackref}
\usepackage{float,parskip}
\usepackage{xparse}

\newfontfamily\chnfont{Source Han Serif SC}
\newfontfamily\jpfont{MS Mincho}
\newcommand{\chn}[1]{{\chnfont #1}}
\newcommand{\jp}[1]{{\jpfont #1}}

\newcommand{\subheader}[1]{\textbf{#1}\quad}

\newenvironment{lverse}
	{
		\par
		\textbf{\LARGE Verse}
		\vspace{.5em}
		\par
	}
	{
		\par
	}
\NewDocumentCommand{\lline}{ m m m O{} }{
	\par
	\begin{minipage}{\textwidth}
		\textbf{\Large Line} \\
		\subheader{Original}#1 \\
		\subheader{Literal}#2 \\
		\subheader{Translation}#3
		\if{#4}{}
		\else
			\begin{quote}
				#4
			\end{quote}
		\fi
	\end{minipage}
	\par
	\vspace{.5em}
	\par
}

\begin{document}

\setlength{\droptitle}{-.5in}
\setlength{\parindent}{0pt}
\setlength{\parskip}{4pt}

\title{
	『ボヘ・スォナ・ラプソディー』 \\
	{\large Japanese Translation for《波西唢呐狂想曲》}
}
\author{王念一}
\maketitle

\begin{section}{Background}
\end{section}

\begin{section}{Translation}
	\begin{lverse}
		\lline
			{妈妈 教我吹唢呐}
			{Mama, teaches me how to play suona}
			{ママ スォナを教えた}
			[
				\textit{Ma}, the most common topic in Chinese curse words.
				To hurt others' feelings, you first greet their mothers. \\
				\textit{Suo'na} (唢呐) is a traditional Chinese musical instrument which is only used in weddings or at funerals---the two biggest events of one's life. \\
				There is an implication that it is not others but his mama who teaches him how to play suona, leading to other people's death, hence the contrast.
			]

		\lline
			{嘟 嘟嘟嘟嘟嘟嘟 一个白字被送走啦}
			{Dur-- du du du du du... One \textit{white name} is sent gone}
			{ドゥ ドゥドゥドゥドゥドゥ どのシラも堕ちた}
			[
				The "du du" is a mimic to the sound of suona. \\
				\textit{White name} is the calling for the fans watching Otto's live stream, since non-课金 users can only send white danmakus. \\
				「送走」means not only "to be sent gone", but also "to be made dead".
			]
		\lline
			{妈妈 我要保护你}
			{Mama, I want to protect you}
			{ママ 守りたい}
			[
				The classic comment goes "might fail to save the mama".
			]
		\lline
			{可拿着唢呐不能拥抱你}
			{But holding a suona, cannot hug you}
			{でも両手は抱くできない}
		\lline
			{妈妈 呜ーー}
			{Mama... Ooh...}
			{ママ ウーー}
		\lline
			{我要去战道 这世界还有好多人有妈妈}
			{I wanna go fawght*, there are so many people in this world who also have their own mama}
			{戦(いくさ)せやう この世界中沢山のママがある}
			[
				In the original 鬼畜 video, the phrase「战斗」(zhan'dou) is pronunced as「战道」(zhan'dao), giving this line a special accent.
				I'm doing a same sound shift from せよ to せやう. \\
				Similarly, "fawght*" stands for "fight".
				The target shall be other people's mamas. \\
				The latter half of the line doesn't quite make sense, since everybody must have a ma.
				But that's where the joke's at. \\
				Classic danmaku goes「离我妈远点」(stay away from my ma).
			]
		\lline
			{这世界需要我 这些妈妈也需要我}
			{This world needs me, so do these mamas need me}
			{必要、僕を このママと親も}
			[I messed around with the order of words so that they rhyme with「我」(wo).]
	\end{lverse}
	\begin{lverse}
		\lline
			{野区 住着鼻三狼}
			{In the jungle, lives \textit{B 3th wolf}}
			{住んで いって三郎}
			[
				「野区」refers to the jungle area in \textit{League of Legends} the videogame, since Otto was once a professional LOL gamer. \\
				「三狼」be the three NPCs living in the jungle in game. \\
				「鼻」is a synophone of the character 「逼」(bi), which essentially means "fucking (adj.)".
			]
		\lline
			{嘟 嘟嘟嘟嘟嘟嘟 唢呐送走鼻大!狼}
			{Dur-- du du du du du... Suona sends gone \textit{B 1st! wolf}}
			{ドゥ ドゥドゥドゥドゥドゥ 払われうだ!ろう}
			[
				The「大」in this line is strongly accented, making a contrast with the following soft「狼」.
				I managed to rhyme it with「だろう」.
				They pretty much sounds the same.
			]
		\lline
			{留下两条炫狗 呜呜闹}
			{Leaves two shiny dog, annoyingly barking "woof woof"}
			{残して犬二匹 ウウナォ}
			[
				「炫狗」is the nickname of Otto's ex-teammate.
				Joke is that after you kill one of the wolves, two remain.
			]
		\lline
			{走位离谱奥利安的费的初 (哇啊啊啊啊啊啊袄!)}
			{Zouwei lipu aoli'an de fei de chu (waaaaaaaao!)}
			{ゾヱ・リプ・オリアン・ド・フェー・ド・チュー (ワァァァァァァォ!)}
			[Really there is no meaning, just mimicking the original \textit{"Gonna leave you all behind and face the truth"} line, but with Otto's spoken phrases.]
		\lline
			{妈妈 呜ーー (米浴说的道理~)}
			{Mama... Ooh... (mi juy zwod ka daaw kid\~)}
			{ママ ウーー (ミー・ユイ・ズォド・カー・ダウ・キド)}
			[
				"mi juy zwod ka daaw kid" comes from a backward played section of record of Otto singing a Cantonese song.
			]
		\lline
			{我不要吹唢呐 我只想现在立刻就操你妈}
			{I don't want to play suona, I only want to right now immediately fuck your ma}
			{スォナ吹(ぶ)きたくない 今すぐに令堂をやれば}
			[
				The「吹き」lines up with the「不」here, on accent, so be it read as ぶ.
				\textit{Grandmother's got joy.}
			]
	\end{lverse}
\end{section}

\end{document}